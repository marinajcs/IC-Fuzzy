\documentclass[12pt]{article}
\usepackage[margin=2.54cm]{geometry}

\usepackage{graphicx}
\usepackage[utf8]{inputenc}
\usepackage{csquotes}

\graphicspath{ {./imgs/} }

\usepackage[spanish]{babel}
\usepackage{hyperref}
\hypersetup{
  colorlinks=true,  % color en lugar de recuadros
  linkcolor=black,  % enlaces internos
  urlcolor=blue   % enlaces externos
}
\usepackage{url}
\usepackage{float}
\usepackage{enumitem}
\usepackage{comment}
\usepackage{wasysym}
\usepackage{amssymb}
\usepackage{multirow}
\usepackage[utf8]{inputenc}
\usepackage[usenames]{color}
\usepackage[document]{ragged2e}
\usepackage[table]{xcolor}
\usepackage{colortbl}
\definecolor{lightgray}{rgb}{0.9, 0.9, 0.9}
\definecolor{black}{rgb}{0, 0, 0}
\renewcommand{\arraystretch}{1.3}
\arrayrulecolor{black}
\setlength{\arrayrulewidth}{0.8pt}

\usepackage[backend=biber,style=apa]{biblatex} % Citación APA
\addbibresource{bibliografia.bib}
\usepackage{csquotes}
\usepackage{setspace}
\onehalfspacing % Iinterlineado a 1.5
\usepackage{titlesec}

% Configuración de la fuente
\usepackage[T1]{fontenc}
\usepackage{newtxtext}

\begin{document}
\begin{titlepage}
	    \centering
	\begin{minipage}{1\textwidth}
		\raisebox{-0.7\height}
		{\includegraphics[width=0.5\textwidth]{UGR-Logo}}
		\raisebox{-0.8\height}{\includegraphics[width=0.49\textwidth]{ETSIIT-logo.png}}
	\end{minipage}
	
	\vspace{1.5cm}
	
	{\Large Máster Universitario en Ingeniería Informática
		
		de la Universidad de Granada \par}
	
	\vspace{1.5cm}
	
	{\Huge \textbf{Práctica de lógica y sistemas difusos}
		
	\vspace{0.6cm}
	
	\textbf{Estudio de un caso práctico}:
			
	Control - Indoor Environment Quality
	 \par}
	
	\vspace{1.5cm}
	
	{\LARGE {Inteligencia Computacional (IC)} \par}
	
	\vspace{1.5cm}
	
	\vfill
	
	{\Large \textbf{Autora} \par}
	{\Large Marina Jun Carranza Sánchez \par}
	\vspace{0.5cm}
    
\end{titlepage}

\newpage
\justifying


\textbf{Resumen:}

Este documento expone un estudio realizado por investigadores de la Universidad de Granada, sobre la implementación de un controlador difuso unificado para mejorar la calidad del ambiente interior (IEQ) en edificios. Este sistema optimiza múltiples parámetros ambientales, como temperatura, humedad, CO2 e iluminación, para maximizar el confort de los ocupantes y minimizar el consumo energético. Además, se compara con otros estudios de controladores de difusión siguiendo otros enfoques. Los resultados muestran que el controlador difuso propuesto mejora la estabilidad y eficiencia en la gestión del IEQ, destacando la importancia de soluciones inteligentes en entornos modernos.

\vline

\textbf{Abstract:}

This document presents a study conducted by researchers from the University of Granada on the implementation of a unified fuzzy controller to improve Indoor Environment Quality (IEQ) in buildings. This system optimizes multiple environmental parameters, such as temperature, humidity, CO2, and lighting, to maximize occupant comfort and minimize energy consumption. Additionally, it compares this approach with other studies on fuzzy controllers following different methodologies. The results show that the proposed fuzzy controller enhances stability and efficiency in IEQ management, highlighting the importance of intelligent solutions in modern environments.

\newpage
\tableofcontents

\newpage
\listoffigures

\newpage
\listoftables

\newpage

\section{Introducción}

La calidad del ambiente interior (Indoor Environment Quatily, IEQ) es un factor relevante en la habitabilidad de edificios tanto residenciales como comerciales. Este concepto no solo abarca parámetros físicos, sino que también se enfoca en el bienestar general de los ocupantes. Los sistemas tradicionales de control no necesariamente optimizan el confort percibido, y con la proliferación de edificios inteligentes y sistemas HVAC avanzados, surge la necesidad de soluciones más centradas en el usuario.

En este trabajo, se realiza un estudio de un caso práctico, donde tres investigadores del Departamento de Ciencias de la Computación e Inteligencia Artificial de la Universidad de Granada (Miguel Molina-Solana, Maria Ros y Miguel Delgado), proponen un controlador difuso unificado que integra diferentes aspectos de la calidad ambiental interior, con el objetivo de optimizar el confort del usuario mientras se reduce el consumo energético \parencite{molina2013unifying}.

\subsection{Motivación y contexto}

La creciente preocupación por el consumo energético, que representa una proporción significativa del gasto global en edificios, junto con la demanda de entornos más cómodos y saludables, ha impulsado el desarrollo de tecnologías más avanzadas de gestión de IEQ. Los sistemas HVAC convencionales no logran responder adecuadamente a la variabilidad de las condiciones ambientales y las preferencias de los usuarios, lo que resalta la necesidad de enfoques más sofisticados.

En este contexto, la lógica difusa ofrece un marco versátil para abordar la complejidad y la interrelación de múltiples parámetros ambientales. El uso de FLC permite no solo controlar eficientemente la temperatura y la humedad, sino también integrar otros factores como la calidad del aire y la iluminación, mejorando así el confort general. Este enfoque, además, facilita la implementación en sistemas ya existentes, proporcionando una capa de control más robusta y adaptativa.


\subsection{Objetivos}

El objetivo principal de este trabajo es diseñar e implementar un sistema de control basado en lógica difusa para optimizar la calidad del ambiente interior (IEQ), integrando múltiples parámetros ambientales con el fin de maximizar el confort de los ocupantes a la vez que se reduce el consumo energético.

A partir de este objetivo principal, se pueden extraer una serie de subobjetivos que vienen recogidos en el Cuadro \ref{tab:subobjetivos}.

\begin{table}[H]
	\centering
	\begin{tabular}{| c | p{9.6cm} |}
		\hline
		\rowcolor{lightgray}
		\textbf{Subobjetivos} & \textbf{Descripción} \\
		\hline
		Optimización del consumo energético & 
		Reducir el consumo energético de los sistemas HVAC mientras se mantienen niveles de adecuados de confort \vspace{0.2cm} \\
		\hline
		Integración de sensores múltiples & 
		Utilizar datos de múltiples sensores (temperatura, humedad, CO2, iluminación) para una evaluación más completa del entorno 
		\vspace{0.2cm} \\
		\hline
		Mejora de la estabilidad del sistema & 
		Reducir las oscilaciones en los parámetros ambientales mediante la implementación de reglas difusas más precisas.
		\vspace{0.2cm} \\
		\hline
		Flexibilidad y escalabilidad & 
		Diseñar un sistema de control que pueda adecuarse fácilmente a diversos entornos y condiciones operativas.
		\vspace{0.2cm} \\
		\hline
		Control predictivo y preventivo & 
		Anticipar cambios en la calidad del ambiente interior y tomar medidas correctivas antes de que los niveles de confort se vean comprometidos.
		\vspace{0.2cm} \\
		\hline
		Personalización del confort &
		Ajustar dinámicamente las condiciones ambientales de acuerdo con las preferencias y actividades de los usuarios
		\vspace{0.2cm} \\
		\hline
	\end{tabular}
	\caption{Subobjetivos generales para mejorar la IEQ.}
	\label{tab:subobjetivos}
\end{table}

Los cuatro primeros subobjetivos de la tabla anterior son de alta prioridad en cualquier implementación de FLC destinada al control de la IEQ, mientras que los dos últimos podrían considerarse funciones o capacidades adicionales, ya que para alcanzarlos es común utilizar enfoques híbridos con otras técnicas avanzadas como el aprendizaje automático.

\newpage
\section{Análisis del problema}

La gestión eficiente de la calidad del ambiente interior (IEQ) influye directamente en la capacidad de garantizar el bienestar y la salud de los ocupantes en espacios cerrados. Sin embargo, la complejidad inherente a las múltiples variables que participan en el IEQ presenta desafíos significativos para los sistemas de control tradicionales.

\subsection{Descripción del problema}

El confort ambiental en interiores depende de la interacción de diversos factores, incluyendo temperatura, humedad relativa, concentración de CO2, iluminación y calidad del aire. Los sistemas de control convencionales, como los controladores PID y On-Off, suelen operar de manera independiente sobre cada variable, sin considerar las interdependencias entre ellas. Esta falta de integración puede conducir a situaciones donde la optimización de un parámetro afecta negativamente a otros, comprometiendo el confort general de los ocupantes. Por ejemplo, un aumento en la ventilación para reducir la concentración de CO2 puede disminuir la temperatura interior, generando incomodidad térmica \parencite{molina2013unifying}.

\subsection{Importancia del problema}

La calidad del ambiente interior (IEQ) es fundamental para el bienestar y la salud de los ocupantes de edificios. Según la Comisión Europea, los edificios son responsables del 40\% del consumo energético de la Unión Europea y del 36\% de las emisiones de gases de efecto invernadero, generadas principalmente durante su construcción, utilización, renovación y demolición \parencite{ComisiónEuropea2020}.

Una gestión ineficiente del IEQ no solo afecta negativamente la salud y el confort de los ocupantes, sino que también contribuye al desperdicio de energía y al aumento de las emisiones de CO2. La Comisión Europea destaca que aproximadamente el 75\% del parque inmobiliario de la UE es ineficiente desde el punto de vista energético, lo que significa que gran parte de la energía consumida se malgasta. Las pérdidas de energía pueden minimizarse mejorando los edificios ya existentes y apostando por soluciones inteligentes y materiales eficientes desde el punto de vista energético para las nuevas construcciones. 

Además, la mejora de la eficiencia energética de los edificios será determinante para el ambicioso objetivo de conseguir la neutralidad en emisiones de carbono establecido para 2050 en el Pacto Verde Europeo. Por lo tanto, abordar la eficiencia energética en la gestión del IEQ es esencial no solo para el bienestar de los ocupantes, sino también para cumplir con los objetivos climáticos y energéticos de la UE.

\subsection{Desafíos en la resolución del problema} 

La gestión de la calidad ambiental interior (CAI) enfrenta varios retos debido a la complejidad inherente del entorno interior. Estos desafíos surgen tanto de la interacción de múltiples variables como de las limitaciones en las infraestructuras y tecnologías disponibles. A continuación, se detallan los principales obstáculos identificados:

\begin{itemize}
	\item \textbf{Complejidad multidimensional}: las variables que afectan el confort interior están interrelacionadas y pueden presentar comportamientos no lineales, lo que dificulta su modelado y control. Dicha complejidad añadida exige una solución integral que permita gestionar simultáneamente todas las variables interdependientes.
	
	\item \textbf{Subjetividad del confort}: el confort es altamente subjetivo y varía entre individuos, lo que complica la creación de un sistema que satisfaga a todos los usuarios. <<Los conceptos de seguridad, limpieza y aislamiento... abarcan mucho más que la concentración de sustancias respirables y no son universales>> \parencite{vargas2005calidad}.
	
	\item \textbf{Condiciones cambiantes}: factores como el clima exterior, la ocupación del espacio y la actividad de los usuarios pueden cambiar constantemente, requiriendo un sistema flexible y adaptativo.
	
	\item \textbf{Optimización energética}: <<El mantenimiento de las condiciones ambientales interiores óptimas se consigue en gran medida a expensas del aumento en el consumo energético>> \parencite{vargas2005calidad}. Esto viene a indicar que existe una tensión inherente entre mejorar el confort y reducir el consumo de energía, y diseñar un sistema que logre ambos objetivos simultáneamente es un reto significativo.
	
	\item \textbf{Integración tecnológica}: incorporar nuevos sistemas de control en infraestructuras existentes sin interrumpir su funcionamiento o requerir grandes inversiones es otro desafío técnico y económico.
	
\end{itemize}

Estos desafíos subrayan la necesidad de una solución innovadora que pueda abordar la complejidad y dinámica del problema de manera eficiente y efectiva.






\newpage
\section{Desarrollo}

En esta sección se va a proporcionar una breve descripción del caso práctico, explicar el diseño detallado del FLC, las pruebas realizadas y los resultados obtenidos, así como las interpretaciones de los mismos. Todas las imágenes en esta sección han sido sacadas del estudio del controlador difuso unificado \parencite{molina2013unifying}, salvo que se indique lo contrario.

\subsection{Descripción del caso práctico}

El estudio presentado propone un controlador difuso unificado para mejorar la gestión de la calidad del ambiente interior (IEQ). Este enfoque busca superar las limitaciones de los sistemas tradicionales de control HVAC, que a menudo son incapaces de manejar de manera eficiente múltiples variables y criterios interrelacionados. 

El caso práctico analizado en la publicación consiste en aplicar el controlador difuso propuesto a una habitación piloto equipada con sensores de temperatura y humedad. 

\subsection{Diseño del controlador difuso}

La propuesta incluye un controlador basado en lógica difusa, recomendado especialmente en aplicaciones donde el modelo matemático exacto del sistema no es conocido, pero su comportamiento puede ser descrito a partir de la experiencia. Este tipo de controlador mejora la flexibilidad ajustándose a los requisitos de confort y puede manejar situaciones críticas de manera más confiable, gracias a reglas basadas en el conocimiento experto.

\subsubsection{Entradas y salidas}

El sistema utiliza cinco sensores, que capturan cinco entradas distintas:

\begin{enumerate}
	\item Temperatura interna ($S_{temp_{indoor}}$)
	\item Temperatura externa ($S_{temp_{outdoor}}$)
	\item Humedad relativa ($S_{RH}$)
	\item Concentración de CO2 ($S_{CO_2}$)
	\item Nivel de iluminación ($S_{light}$)
\end{enumerate}

Las salidas son cuatro, que se corresponden a tres actuadores:
\begin{enumerate}
	\item Programa del aire acondicionado ($A_{Air}$): caliente, frío, seco.
	\item Nivel de temperatura ($A_{temp_{level}}$): bajar, mantener, subir.
	\item Nivel de iluminación ($A_{light}$): bajo, medio y alto.
	\item Nivel del humedad ($A_{h_{level}}$): apagado, bajo, estándar, alto, continuo.
\end{enumerate}

Estos actuadores regulan el confort del ambiente interior al influir en el sistema HVAC. Las decisiones sobre estas salidas se toman con base en las lecturas de los cinco sensores y se ajustan dinámicamente según las reglas difusas definidas en el sistema

\subsubsection{Funcionamiento del controlador}

El FLC se basa en un motor de inferencia, que procesa las entradas tras la fuzzificación y genera salidas mediante un proceso de defuzzificación. Esto permite ajustar el sistema para lograr el confort del usuario.

\begin{figure}[H]
	\centering
	\includegraphics[width=0.85\textwidth]{imgs/arquitectura-FLC.JPG}
	\caption{Arquitectura PID y controlador fuzzy.}
	\label{fig:arquitectura-FLC}
\end{figure}

La \autoref{fig:arquitectura-FLC} muestra la arquitectura del FLC, diseñado para regular diferentes variables ambientales mediante un conjunto de sensores y actuadores. Los sensores monitorean parámetros clave, como la temperatura interna y externa, la humedad y la concentración de CO2. Estas lecturas se envían al controlador difuso, que analiza los datos en combinación con una base de conocimiento. Esta base de conocimiento contiene reglas y relaciones definidas por expertos que permiten al sistema tomar decisiones informadas.

El controlador difuso procesa las entradas y determina las acciones necesarias para mantener las condiciones óptimas dentro del entorno. Las decisiones se transmiten a una serie de controladores específicos para cada variable, como el controlador de temperatura, el de humedad y el de CO2. Finalmente, estos controladores ajustan los actuadores correspondientes para implementar las acciones recomendadas. De esta manera, el sistema asegura un ambiente confortable y eficiente, gestionando las interdependencias entre las distintas variables

\vspace{0.4cm}

La base de conocimiento del FLC incluye:
\begin{enumerate}
	\item \textbf{Funciones de pertenencia}: se utilizan funciones trapezoidales para describir las etiquetas lingüísticas (Bajo, Medio, Alto) que serán empleadas para el proceso de fuzzificación.
	\begin{figure}[H]
		\centering
		\includegraphics[width=0.55\textwidth]{imgs/function.JPG}
		\caption{Función de pertenencia.}
		\label{fig:function}
	\end{figure}
	
	En la \autoref{fig:function}: $L$ puede tomar el valor de Bajo, Medio y Alto; $a$ y $d$ son los puntos extremos de la función de pertenencia trapezoidal; $b$ y $c$, los puntos máximos de la función de pertenencia trapezoidal; y $x_i$ es el $i$-ésimo sensor.
	
	Las Figuras \ref{fig:membership-functions-input} y \ref{fig:membership-functions-output} muestran las funciones de pertenencia utilizadas en el sistema de control difuso para las entradas y las salidas, respectivamente. Estas funciones son necesarias para convertir los valores numéricos (\textit{crisp}) de los sensores en términos lingüísticos (bajo, medio, alto) que el sistema difuso pueda interpretar y procesar.
	
	\begin{figure}[H]
		\centering
		\includegraphics[width=0.8\textwidth]{imgs/membership-functions-input.JPG}
		\caption{Función de pertenencia de las variables de entrada.}
		\label{fig:membership-functions-input}
	\end{figure}
	
	\begin{figure}[H]
		\centering
		\includegraphics[width=0.8\textwidth]{imgs/membership-functions-output.JPG}
		\caption{Función de pertenencia de las variables de salida}
		\label{fig:membership-functions-output}
	\end{figure}
	
	\item \textbf{Reglas difusas}: conjunto de reglas \textit{IF-THEN} derivadas del conocimiento experto. Ejemplos:
	\begin{itemize}
		\item Si la temperatura interna es media y la externa es alta, entonces se mantiene la temperatura y se cambia el humidificador al nivel estándar.
		\item Si la humedad y la temperatura interna son bajas, entonces se pone el aire acondicionado en caliente, el humidificador en nivel alto y se sube la temperatura.
	\end{itemize}
	
	Se ha utilizado un total de 17 reglas, que dan lugar a diversos escenarios al combinar las posibles entradas del sistema. A continuación se muestra un subconjunto de dichas reglas:
	
	\begin{figure}[H]
		\centering
		\includegraphics[width=0.9\textwidth]{imgs/set-of-rules.JPG}
		\caption{Conjunto de reglas.}
		\label{fig:set-of-rules}
	\end{figure}
	
\end{enumerate}

El motor de inferencia aplica el método \textit{Mamdani Max-Min} para combinar reglas y calcular las salidas. 

Como se ha visto antes, cada regla difusa tiene una condición (\textit{IF}) y una conclusión (\textit{THEN}). Para combinar las condiciones de entrada (los valores medidos por los sensores) con los antecedentes de las reglas, se usa el operador \textit{min} (intersección). Esto calcula el grado de cumplimiento de la regla considerando las funciones de pertenencia de las entradas (en este caso, las funciones trapezoidales que traducen los valores crisp a difusos).

Luego, para combinar las salidas de varias reglas activadas, se usa el operador \textit{max} (unión), generando así el conjunto difuso de salida que representa las conclusiones de todas las reglas activadas.

\begin{figure}[H]
	\centering
	\includegraphics[width=0.50\textwidth]{imgs/mamdani.JPG}
	\caption{Método Mamdani Max-Min.}
	\label{fig:mamdani}
\end{figure}

En la \autoref{fig:mamdani}: $x$ son las mediciones de los sensores de entrada, $\alpha_i$ es el grado en que una entrada dada satisface la condición de la $i$-ésima regla ($R_i$), y $\mu_{output_i}$, la agregación de los conjuntos difusos de salida de todas las reglas para $output_i$.

Dado que el enfoque empleado es \textit{Max-Min}, se puede asumir que se ha utilizado el \textit{Modo A - FATI} (\textit{First Aggregate Then Infer}), que como su nombre indica, consiste en primero agregar los conjuntos difusos individuales inferidos, y después defuzzificar para obtener un valor preciso, manteniendo toda la información difusa hasta el último paso \parencite{peregrin2000integracion}.

Para el proceso de defuzzificación, en el estudio no se concreta de manera explícita el método utilizado, pero es muy probable que se haya empleado el método \textit{centroide}, también conocido como centro de gravedad, ya que es uno de los más comunes asociados al método Mamdani tradicional. 

Esta técnica calcula el punto medio del área bajo la curva del conjunto difuso resultante. Matemáticamente (\autoref{fig:centroide}), se obtiene como el cociente entre: la suma ponderada de todos los valores de salida posibles ($z$) en el rango del conjunto difuso (cuánto contribuye cada valor $z$ al equilibrio general) y la suma total de los grados de pertenencia ($C(z)$) en el conjunto difuso.

\begin{figure}[H]
	\centering
	\includegraphics[width=0.33\textwidth]{imgs/centroide.JPG}
	\caption{Fórmula para calcular el centroide \parencite{klir1996fuzzy}.}
	\label{fig:centroide}
\end{figure}

\subsection{Pruebas en simuladores y entornos reales}

Se han realizado pruebas del FLC tanto en simuladores como en entornos físicos reales.

\subsubsection{Configuración del simulador}

Se desarrolló un simulador con una interfaz web para probar el controlador difuso que va a gestionar los parámetros de confort ambiental en interiores.

Dicho simulador permite definir rangos de confort, ingresar valores de las condiciones actuales del ambiente (temperatura, iluminación, humedad y concentración de CO2) y obtener las acciones necesarias para mantener dichos parámetros dentro de los rangos deseados.

El simulador contiene tres paneles, asociados a tres elementos distintos:
\begin{enumerate}
	\item \textbf{Los rangos de confort}: este panel permite definir los intervalos de confort que utilizará el controlador para sus reglas difusas. Por ejemplo, se pueden establecer valores específicos para las etiquetas (baja, media y alta) de humedad o temperatura.
	
	\begin{figure}[H]
		\centering
		\includegraphics[width=0.6\textwidth]{imgs/simulator-comfort-params.JPG}
		\caption{Panel de definición de intervalos de confort.}
		\label{fig:simulator-comfort-params}
	\end{figure}
	
	\item \textbf{Los valores de entrada}: encargado de registrar los datos de las condiciones de interior actuales, simulando los valores que los sensores reales proporcionarían en un entorno físico real. Estas entradas corresponden a las variables que figuran en los antecedentes de las reglas difusas.
	
	\begin{figure}[H]
		\centering
		\includegraphics[width=0.6\textwidth]{imgs/simulator-panel-inputs.JPG}
		\caption{Panel de definición de valores de entrada.}
		\label{fig:simulator-panel-inputs}
	\end{figure}
	
	\item \textbf{Los valores de salida}: donde se disponen las acciones sugeridas por el controlador para ajustar el ambiente, y así poder reajustar las condiciones actuales dentro de los intervalos de confort definidos en el primer panel.
	
	\begin{figure}[H]
		\centering
		\includegraphics[width=0.6\textwidth]{imgs/simulator-actions-output.JPG}
		\caption{Panel de definición de valores de salida.}
		\label{fig:simulator-actions-output}
	\end{figure}
	
	El formato de las salidas son acciones concretas con un grado de aplicación especificado, en el supuesto de que deba aplicarse. A continuación se muestran algunos ejemplos de salidas:
	
	\begin{itemize}
		\item Si la iluminación interior y exterior es baja, se sugiere activar la iluminación artificial.
		\item Si la humedad relativa interior es del 19\%, por debajo del umbral de confort del 30\%, se recomienda activar el humidificador.
		\item Si la temperatura interior es baja y la exterior también, se instruye cerrar las ventanas y aumentar la temperatura usando el sistema HVAC.
	\end{itemize}
	
\end{enumerate}

Los resultados obtenidos concuerdan con las previsiones de los expertos en energía que participaron en el proyecto. Esto demuestra que las reglas implementadas reflejan adecuadamente las estrategias recomendadas para mantener un entorno confortable en interiores. 

Además, el simulador se convirtió en una herramienta útil para la mejora continua, ya que permitió a los expertos realizar ajustes y optimizaciones de manera iterativa. Pues inicialmente, se partió del conjunto de reglas definidas por los expertos, basándose en su conocimiento y experiencia previa; sin embargo, al observar los resultados generados por el simulador en distintos escenarios, pudieron identificar áreas de mejora, logrando así mejorar la precisión del controlador y su capacidad de adaptación a situaciones cambiantes. 

En resumen, el simulador no solo sirvió para validar las reglas iniciales, sino que también actuó como un laboratorio virtual donde los expertos podían experimentar con diferentes configuraciones. Esto permitió un desarrollo progresivo hacia un sistema más robusto, eficiente y alineado con los objetivos de confort y sostenibilidad energética.

\subsubsection{Configuración experimental}

Para complementar las pruebas del simulador previamente mencionado, el controlador difuso fue sometido a una serie de tests en un entorno real, concretamente, en una habitación equipada con varios sensores.

La habitación experimental utilizada, con unas dimensiones de 5.2m x 5.2m x 2.5m, tenía las siguientes características:
\begin{itemize}
	\item Una puerta y una ventana ubicadas en posiciones opuestas.
	\item Un sistema HVAC con calefacción, ventilación y aire acondicionado.
	\item Dos sensores para medir la temperatura y humedad, respectivamente (1).
	\item La ausencia de muebles y ocupantes durante el período de pruebas (2).
\end{itemize}

(1) De esta manera, aunque no pudieron obtenerse resultados en función de todos los parámetros implicados en la IEQ debido a limitaciones técnicas, las dos variables que sí pudieron tomarse en cuenta proporcionaron resultados interesantes.

(2) Para así poder eliminar grandes interferencias externas que pudieran alterar las mediciones; aunque al final la puerta se abrió ocasionalmente, con la entrada breve de algunas personas, lo que terminó generando pequeñas perturbaciones que fueron consideradas en el análisis.

\vspace{0.4cm}

En cuanto a la recopilación y almacenamiento de datos, se tomó el mes de diciembre como período de pruebas, capturando los datos medidos cada 15 minutos. Estas mediciones, que incluían información sobre la temperatura y humedad dentro de la habitación, se exportaron a una base de datos para su posterior análisis y comparación.

\subsection{Resultados del estudio}

Para este subapartado, se han tenido en cuenta los resultados que se obtuvieron en las pruebas experimentales en el período de diciembre de 2012.

Para analizar el rendimiento y utilidad de la propuesta, se realiza una comparación de los resultados obtenidos en la sala piloto cuando se ejecutan dos tipos distintos de controladores: el controlador difuso diseñado y un controlador reactivo más tradicional.

En el caso de la temperatura (\autoref{fig:graph-temperatures-december}), se observa que el controlador difuso, representado en la zona gris del gráfico, logra mantener la temperatura en el rango de confort de manera constante y con mínima variación. Esto significa que el controlador difuso no solo ajusta la temperatura dentro de los límites deseados, sino que también evita grandes fluctuaciones, proporcionando un ambiente estable. En cambio, durante el periodo en blanco, cuando operaba el controlador reactivo, aunque la temperatura también se mantenía en el rango de confort, las oscilaciones eran notoriamente más pronunciadas, lo que implica un ajuste menos preciso y estable.

\begin{figure}[H]
	\centering
	\includegraphics[width=0.95\textwidth]{imgs/graph-temperatures-december.JPG}
	\caption{Valores de la temperatura recogidos en diciembre de 2012.}
	\label{fig:graph-temperatures-december}
\end{figure}

Para la humedad relativa (\autoref{fig:graph-humidity-december}), aunque la estabilidad es menos clara que en el caso de la temperatura, el controlador difuso también demuestra un mejor rendimiento. En la zona gris, donde el controlador difuso estaba activo, las oscilaciones en la humedad son menores, manteniéndose dentro del rango de confort de forma más controlada. Esto contrasta con el periodo inicial, donde el controlador reactivo produce mayores fluctuaciones, lo que sugiere que el controlador difuso es más efectivo en estabilizar tanto la temperatura como la humedad relativa.

\begin{figure}[H]
	\centering
	\includegraphics[width=0.95\textwidth]{imgs/graph-humidity-december.JPG}
	\caption{Valores de la humedad recogidos en diciembre de 2012.}
	\label{fig:graph-humidity-december}
\end{figure}

En resumen, estos resultados indican que el controlador difuso no solo es eficaz en mantener las condiciones de confort (temperatura y humedad) dentro de los límites deseados, sino que además logra hacerlo con menos variaciones, creando un ambiente más estable en comparación con el controlador reactivo.




\newpage
\section{Estudios y enfoques alterativos}

A continuación, se han evaluado dos enfoques alternativos de control difuso de la calidad del ambiente interior, para más adelante hacer un estudio comparativo de los tres.

\subsection{Control fuzzy de sistemas HVAC optimizado con algoritmos genéticos}

Este trabajo, realizado por un investigador de la Universidad de Jaén (Rafael Alcalá) y cuatro de la Universidad de Granada (José M. Benitez, Jorge Casillas, Oscar Cordón y Raúl Pérez) combina control difuso con algoritmos genéticos para optimizar los sistemas HVAC, mejorando la eficiencia energética y el confort \parencite{alcala2003fuzzy}.

El estudio se centró en el control de sistemas HVAC en dos sitios de prueba reales, con el objetivo de optimizar tanto el rendimiento energético como las condiciones de confort interior. Estos sitios incluían un centro de investigación en Francia y una instalación privada. Cada uno tenía características específicas, como sistemas HVAC de diferente configuración, lo que presentaba un desafío adicional para diseñar un controlador adaptable y eficiente. Los expertos proporcionaron modelos térmicos detallados para ambos sitios, ajustados a las condiciones climáticas y de ocupación específicas de cada temporada.

\subsubsection{Diseño del controlador difuso}

El controlador difuso fue diseñado para manejar múltiples criterios, como el confort térmico, la calidad del aire interior y el consumo de energía, usando una base de reglas construida con la experiencia de expertos. Además, se utilizaron funciones de pertenencia triangulares para simplificar el proceso de inferencia y mejorar la manejabilidad del sistema.

A diferencia del estudio principal que se trata en este documento, este proyecto también integró algoritmos genéticos (AG) para afinar automáticamente los parámetros del controlador difuso. Estos AG optimizaron las bases de conocimiento ajustando las funciones de pertenencia para mejorar el rendimiento del sistema bajo diferentes condiciones operativas. El método propuesto recibió el nombre de \textit{Weighted Multi-Criteria Steady-State Genetic Algorithm} (WMC-SSGA), cuyo comportamiento viene reflejado en la \autoref{fig:flowchart-ga}.

\begin{figure}[H]
	\centering
	\includegraphics[width=0.45\textwidth]{imgs/flowchart-ga.JPG}
	\caption{Ddiagrama de flujo del algoritmo WMC-SSGA \parencite{alcala2003fuzzy}.}
	\label{fig:flowchart-ga}
\end{figure}

No se va a entrar en detalle con el funcionamiento del algoritmo genético, pero se podría resumir de la siguiente manera:

El WMC-SSGA comienza con la generación de una población inicial dentro de unos límites. Estas soluciones potenciales codifican parámetros para el controlador difuso, que serán evaluadas por el algoritmo basándose en una función objetivo multicriterio ponderada (con indicadores como minimizar el consumo energético y mantener los niveles de confort). Si no hay suficiente mejora, se ajustan los límites, y después, se seleccionan las mejores soluciones (padres), que se combinan y modifican para crear nuevas soluciones. Este proceso se repite de forma iterativa, evaluando y mejorando las soluciones hasta llegar a un límite o encontrar una solución lo suficientemente buena.

El método anterior contrasta con el FLC unificado, que carece de capacidades de autoajuste y depende únicamente de reglas definidas por expertos.


El controlador difuso presenta una estructura jerárquica, diseñada para procesar múltiples variables y tomar decisiones complejas optimizando el rendimiento del sistema HVAC. Esta estructura se organiza en módulos jerárquicos, donde cada nivel se encarga de tareas específicas y alimenta al siguiente. Se muestra un ejemplo en la \autoref{fig:fuzzy-genetics-functions}.

En la primera capa, se procesan variables básicas como las entradas del índice de control térmico (\textit{PMV}), el valor del PMV anterior (\textit{PMV t-1}), la diferencia entre la temperatura exterior y la interior (\textit{Tout-Tin}) y la posición de la válvula antes del ajuste (\textit{Valve old position}). 

Las flechas entre módulos indican el flujo de la información, señalando cómo los módulos de niveles inferiores sirven para calcular valores que son utilizados como entradas por los niveles superiores. Por ejemplo, partiendo de la \textit{PMV} y \textit{PMV t-1}, se obtiene la preferencia térmica (\textit{Thermal preference}) basada en las condiciones actuales del ambiente. Y esta variable, que pertenece a la segunda capa, puede utilizarse con la \textit{Tout-Tin} para obtener calor requerido (\textit{Required heat}). Este último es necesario en la tercera capa para, junto con la anterior posición de la válvula y la prioridad térmica/energética, poder calcular tanto la nueva posición de la válvula como la velocidad del ventilador del sistema HVAC.

\begin{figure}[H]
	\centering
	\includegraphics[width=0.95\textwidth]{imgs/fuzzy-genetics-functions.JPG}
	\caption{Diagrama de módulos jerárquicos y funciones de pertenencia \parencite{alcala2003fuzzy}.}
	\label{fig:fuzzy-genetics-functions}
\end{figure}

Las optimizaciones fruto del algoritmo genético vienen reflejadas en la \autoref{fig:fuzzy-genetics-functions} mediante las líneas negras de las gráficas, en contraste con las grises, que representan la configuración original del sistema. Tras la optimización, las funciones muestran ligeros desplazamientos y ajustes en sus picos y bordes, adaptándose mejor a los datos del modelo, ya que estos cambios buscan maximizar la precisión y robustez del sistema.

Asimismo, a diferencia del estudio anterior, en este se ha seguido el \textit{Modo B - FITA} (\textit{First Infer Then Aggregate}). <<En este modo de trabajo, se considera individualmente la contribución de cada conjunto difuso inferido y la acción precisa final se obtiene mediante algún tipo de operación efectuada sobre un valor preciso
característico obtenido a partir de cada conjunto difuso individual. De este modo, se
evita el calculo del conjunto difuso final, lo que ahorra una gran cantidad de
tiempo de cálculo>> \parencite{peregrin2000integracion}. Dicho en otras palabras, cada conjunto difuso resultante de una regla se defuzzifica individualmente, y posteriormente, se utiliza una técnica para combinar los valores crisp obtenidos y así poder producir la salida final.

Para la defuzzificación, en lugar del cálculo del centroide, se ha empleado la técnica \textit{MOM} (\textit{Mean of Maxima}) o media de los máximos, en el que se seleccionan todos los puntos del conjunto difuso de salida donde la función de pertenencia alcanza su valor máximo, para luego calcular el promedio aritmético de esos puntos máximos.

\begin{figure}[H]
	\centering
	\includegraphics[width=0.33\textwidth]{imgs/mom.JPG}
	\caption{Fórmula de Mean of Maxima (MOM) \parencite{klir1996fuzzy}.}
	\label{fig:mom}
\end{figure}

En \autoref{fig:mom}, se considera un conjunto $M$ de valores crisp ($z_k$) que tienen el grado de pertenencia máximo dentro del conjunto difuso $C$. 

\subsubsection{Pruebas}

Se llevaron a cabo experimentos tanto en simulaciones como en entornos reales. Los primeros permitieron evaluar durante períodos de 10 días las distintas configuraciones del controlador bajo condiciones climáticas específicas; mientras que los segundos proporcionaron validación en condiciones más realistas. Los resultados se compararon con un controlador convencional On-Off y con versiones iniciales no optimizadas del FLC para medir la mejora alcanzada.

\subsubsection{Resultados y conclusiones}

Al igual que el controlador difuso unificado, los resultados de este estudio mostraron mejoras significativas en comparación con los controladores tradicionales.

En términos de eficiencia energética, en este estudio se lograron ahorros de hasta un 20\% en algunos casos gracias al uso de algoritmos genéticos, mientras que los parámetros de confort se mantuvieron dentro de los rangos deseados con fluctuaciones mínimas. Estos resultados sugieren que la integración de técnicas avanzadas de optimización puede potenciar aún más la eficacia de los controladores difusos.

\subsection{Sistema de inferencia difusa centrado en la evaluación de IEQ}

El sistema propuesto por Karol Jabłonski y Tomasz Grychowski, miembros de una universidad polaca, se diseñó para evaluar las condiciones ambientales interiores en un edificio mediante un sistema de sensores y un módulo de inferencia difusa. Similar al primer estudio, este sistema se enfocó en múltiples parámetros como temperatura, humedad y CO2, pero también integró otros factores como iluminación, ruido y olores, proporcionando una evaluación más completa del confort interior \parencite{jablonski2018fuzzy}.

El estudio no describe un controlador difuso en el sentido clásico (es decir, un sistema que toma decisiones para regular un proceso dinámico), sino más bien un sistema de inferencia difusa diseñado para evaluar y calificar el confort ambiental en interiores. Aunque emplea lógica difusa y comparte componentes clave de un controlador difuso (como funciones de pertenencia, reglas difusas y defuzzificación), su propósito es diferente.

Dicho esto, el sistema podría evolucionar hacia un controlador difuso si sus salidas fueran utilizadas para ajustar automáticamente dispositivos como sistemas HVAC.

\subsubsection{Diseño del sistema difuso}

Este estudio adoptó un sistema híbrido de inferencia difusa para procesar múltiples índices, utilizando una estructura modular con subsistemas independientes para manejar los distintos parámetros.

La \autoref{fig:fuzzy-inference-system-diagram} representa el funcionamiento del sistema de inferencia difusa, que se distingue por su enfoque integral, al combinar múltiples parámetros ambientales mediante subsistemas especializados que contribuyen a una evaluación global del confort. 

Las variables monitoreadas incluyen temperatura del aire, temperatura de globo negro, humedad relativa, densidad de CO2, iluminación, ruido y olores, que se procesan a través de módulos de inferencia difusa para estimar diferentes índices de confort. Cada uno de estos índices (confort térmico, frescura del aire y fatiga) es calculado de forma independiente en tres subsistemas, y luego integrado en un módulo principal (confort general) que genera una evaluación global del confort percibido.

Así, a diferencia de los otros dos sistemas que se centran en un conjunto más reducido de variables, este enfoque incluye factores adicionales como olores y ruido, que pueden tener un impacto significativo en la percepción del confort. Además, el sistema consta de 92 reglas difusas, distribuidas entre 4 subsistemas principales, lo que contrasta con un enfoque unificado, que habría requerido hasta 2400 reglas para cubrir todas las combinaciones posibles de entradas.

\begin{figure}[H]
	\centering
	\includegraphics[width=0.75\textwidth]{imgs/fuzzy-inference-system-diagram.JPG}
	\caption{Diagrama del sistema de inferencia difusa \parencite{jablonski2018fuzzy}.}
	\label{fig:fuzzy-inference-system-diagram}
\end{figure}

En este estudio se han utilizado funciones de pertenencia triangulares para las salidas y trapezoidales para algunas entradas. Se muestra un ejemplo en la \autoref{fig:fuzzy-2-membership-lassitude-freshness}. 

\begin{figure}[H]
	\centering
	\includegraphics[width=0.5\textwidth]{imgs/fuzzy-2-membership-lassitude-freshness.JPG}
	\caption{Funciones de pertenencia de la fatiga y frescura \parencite{jablonski2018fuzzy}.}
	\label{fig:fuzzy-2-membership-lassitude-freshness}
\end{figure}

Además, este sistema destaca por implementar un módulo de inferencia en \textit{LabVIEW}, diseñando desde cero los bloques de fuzzificación, inferencia y defuzzificación, mientras que en otros estudios se utilizaron herramientas estándar o adaptaciones de controladores PID.

\subsubsection{Pruebas}

Las pruebas en este sistema se realizaron en un entorno real, con mediciones de confort en diferentes condiciones, incluyendo variaciones en ocupación y ventilación en múltiples escenarios. 

Durante las pruebas, se realizaron encuestas para validar los índices calculados por el sistema, permitiendo alinear las percepciones humanas con las salidas del sistema y obteniendo una alta concordancia entre ambas. El enfoque anterior no está presente en los otros dos estudios, donde la validación se centró más en la comparación con sistemas tradicionales o en el análisis de la eficiencia energética.

\subsubsection{Resultados y conclusiones}

En la \autoref{fig:fuzzy-inference-system-results}, se comparan las salidas del sistema difuso con encuestas subjetivas en los cuatro aspectos del confort ambiental definidos por los subsistemas. Los resultados muestran que el sistema sigue tendencias similares a las percepciones humanas, con picos y descensos relacionados con cambios en las condiciones ambientales, aunque es más continuo y menos variable que las encuestas. Esto valida la capacidad del sistema para evaluar objetivamente el confort, reflejando correctamente los efectos de factores como temperatura, iluminación y concentración de CO2.

\begin{figure}[H]
	\centering
	\includegraphics[width=0.76\textwidth]{imgs/fuzzy-inference-system-results.JPG}
	\caption{Resultados de las encuestas y medidas del sistema \parencite{jablonski2018fuzzy}.}
	\label{fig:fuzzy-inference-system-results}
\end{figure}

Este enfoque destaca por la detección de eventos específicos como la presencia de alimentos o cambios en la ocupación. De esta forma, este sistema amplía el rango de variables monitoreadas y evaluadas, proporcionando una evaluación más centrada en todo aquello que define IEQ.

Como ventajas adicionales, dividir un sistema difuso en subsistemas reduce drásticamente la complejidad al evitar el crecimiento exponencial ($n^m$, siendo $n$ es el número de valores de entrada y $m$ el número de entradas) de reglas que ocurriría en un sistema unificado, donde cada entrada y sus combinaciones requieren cobertura. Esto hace que el diseño sea más manejable y ajustable. Además, la modularidad permite mejorar o expandir cada subsistema de forma independiente, sin afectar al resto; por ejemplo, al añadir un nuevo sensor solo se ajusta el subsistema correspondiente. También mejora la eficiencia computacional, ya que cada subsistema realiza cálculos más simples, permitiendo respuestas rápidas a los cambios ambientales.

Aunque dividir un sistema difuso en subsistemas ofrece claras ventajas, también puede presentar algunas desventajas en comparación con el enfoque unificado. Una de las principales limitaciones es la pérdida de interacción directa entre todas las variables de entrada. Al segmentar las entradas en módulos, las interacciones complejas que podrían influir en el resultado general pueden no ser capturadas de manera precisa, ya que cada subsistema procesa solo un subconjunto de las variables. Además, requiere una fase adicional de integración para combinar las salidas de los subsistemas en un resultado global, lo que puede introducir incertidumbre o simplificaciones adicionales. Finalmente, aunque se reduce la complejidad computacional, un sistema unificado podría ser más eficiente en aplicaciones donde el hardware es suficientemente potente, ya que permite procesar todas las combinaciones posibles en una sola etapa sin necesidad de combinar resultados.

\subsection{Comparaciones entre los tres estudios}

Las comparaciones entre FLCs diseñados se han realizado en base a cinco aspectos generales:

\begin{table}[H]
	\centering
	\renewcommand{\arraystretch}{1.5}
	\begin{tabular}{|p{2.5cm}|p{4cm}|p{4cm}|p{4cm}|}
		\hline
		\rowcolor{lightgray}
		\textbf{Aspecto} & \textbf{Controlador difuso unificado} & \textbf{Controlador con algoritmos genéticos} & \textbf{Sistema de inferencia difusa} \\ \hline
		
		\textbf{Variables controladas} & 
		Temperatura, humedad, CO2, iluminación & 
		Temperatura, humedad, con optimización dinámica & 
		Temperatura, humedad, CO2, iluminación, ruido, olores \\ \hline
		
		\textbf{Enfoque metodológico} & 
		Controlador difuso clásico con reglas definidas por expertos. FLC supervisa PIDs para HVAC & 
		Integración de algoritmos genéticos para optimizar automáticamente reglas y funciones de pertenencia & 
		Arquitectura modular con subsistemas independientes. Implementación en LabVIEW \\ \hline
		
		\textbf{Pruebas y validación} & 
		Pruebas en una habitación piloto controlada, comparando con controlador reactivo & 
		Simulaciones y pruebas en múltiples entornos reales bajo diferentes condiciones estacionales & 
		Pruebas en entornos ocupados reales, correlacionando resultados con encuestas de usuarios \\ \hline
		
		\textbf{Objetivo principal} & 
		Optimizar el confort interior y la eficiencia energética en sistemas HVAC & 
		Equilibrio dinámico entre confort y eficiencia energética & 
		Evaluación integral del confort interior basado en múltiples índices \\ \hline
		
		\textbf{Aplicación recomendada} & 
		Escenarios con reglas de confort bien definidas, simplicidad y robustez & 
		Entornos donde se requiere máxima eficiencia energética y adaptabilidad & 
		Edificios donde la percepción humana del confort es crítica \\ \hline
	\end{tabular}
	\caption{Comparación de tres estudios sobre controladores difusos para IEQ.}
	\label{tab:comparacion}
\end{table}

En conjunto, estos tres enfoques destacan la flexibilidad de los controladores difusos y su capacidad para adaptarse a distintas necesidades y restricciones operativas.

La elección entre estos enfoques dependerá del contexto de aplicación y los objetivos específicos, como si se busca mayor integración en sistemas existentes o una evaluación más detallada de los parámetros de confort.

\newpage
\section{Conclusiones y trabajos futuros}

En conclusión, este

\subsection{Consecución de objetivos}

Se recupera la tabla de objetivos de la sección \textit{1.2.} (Cuadro \ref{tab:subobjetivos}) para marcar aquellos que han sido alcanzados.

\subsection{Trabajos futuros}

Cabe mencionar que la configuración de la habitación experimental descrita en la sección \textbf{\textit{3.3.2.}}, aunque ideal para este tipo de experimentos al permitir un mayor control sobre las variables, no es representativa de un entorno habitado típico. 

En futuros experimentos, se podría introducir factores adicionales, como la presencia de personas y muebles, para evaluar el desempeño del controlador en condiciones más realistas y complejas.



\newpage
\section{Bibliografía}
\defbibheading{empty}{}
\renewcommand{\bibitemsep}{1em}
\printbibliography[heading=empty]

\end{document}