\section{Estudios y enfoques alterativos}


\subsection{Control fuzzy de sistemas HVAC optimizado con algoritmos genéticos}
Este trabajo combina control difuso con algoritmos genéticos para optimizar los sistemas HVAC, mejorando la eficiencia energética y el confort. \parencite{alcala2003fuzzy}

Relevancia: Es particularmente relevante como alternativa, ya que añade una capa de optimización automática a través de algoritmos genéticos. Esto permite ajustar de forma automática las reglas difusas y los parámetros, lo cual es útil en sistemas complejos con múltiples criterios de control como el de este caso.


\subsection{Sistema de inferencia difusa centrado en la evaluación de calidad ambiental}
: \parencite{jablonski2018fuzzy}

Objetivo: Diseñar un sistema que evalúe el confort ambiental considerando múltiples parámetros como temperatura, humedad, calidad del aire y ruido.
Implementación: Utiliza un sistema de microcontroladores con sensores y una aplicación de PC para la inferencia difusa.
Resultados: Este enfoque se enfoca en medir y analizar simultáneamente múltiples parámetros para evaluar el confort global y específico (como frescura del aire y confort térmico).
Ventajas: Ofrece un sistema integral para evaluar y analizar el confort ambiental en edificios inteligentes, con capacidad de identificar causas de incomodidad y posibles fallos en sistemas HVAC.


\subsection{Comparación y resultados interesantes}
Sistema de Inferencia Difusa (2017): Proporciona una evaluación más completa del confort ambiental, integrando un amplio rango de factores y enfocándose en un análisis cualitativo basado en estándares y percepción humana.
Controlador Difuso Unificado (2013): Está más orientado a la optimización energética y la integración eficiente de subsistemas mediante ajustes dinámicos y adaptativos.
Ambos enfoques son prometedores, pero el primero se centra más en la evaluación global del confort, mientras que el segundo apunta a un control eficiente y coordinado de múltiples sistemas dentro de un entorno.


La elección entre estos enfoques dependerá del contexto de aplicación y los objetivos específicos, como si se busca mayor integración en sistemas existentes o una evaluación más detallada de los parámetros de confort