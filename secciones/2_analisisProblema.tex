\section{Análisis del problema}

La gestión eficiente de la calidad del ambiente interior (IEQ) influye directamente en la capacidad de garantizar el bienestar y la salud de los ocupantes en espacios cerrados. Sin embargo, la complejidad inherente a las múltiples variables que participan en el IEQ presenta desafíos significativos para los sistemas de control tradicionales.

\subsection{Descripción del problema}

El confort ambiental en interiores depende de la interacción de diversos factores, incluyendo temperatura, humedad relativa, concentración de CO2, iluminación y calidad del aire. Los sistemas de control convencionales, como los controladores PID y On-Off, suelen operar de manera independiente sobre cada variable, sin considerar las interdependencias entre ellas. Esta falta de integración puede conducir a situaciones donde la optimización de un parámetro afecta negativamente a otros, comprometiendo el confort general de los ocupantes. Por ejemplo, un aumento en la ventilación para reducir la concentración de CO2 puede disminuir la temperatura interior, generando incomodidad térmica \parencite{molina2013unifying}.

\subsection{Importancia del problema}

La calidad del ambiente interior (IEQ) es fundamental para el bienestar y la salud de los ocupantes de edificios. Según la Comisión Europea, los edificios son responsables del 40\% del consumo energético de la Unión Europea y del 36\% de las emisiones de gases de efecto invernadero, generadas principalmente durante su construcción, utilización, renovación y demolición \parencite{ComisiónEuropea2020}.

Una gestión ineficiente del IEQ no solo afecta negativamente la salud y el confort de los ocupantes, sino que también contribuye al desperdicio de energía y al aumento de las emisiones de CO2. La Comisión Europea destaca que aproximadamente el 75\% del parque inmobiliario de la UE es ineficiente desde el punto de vista energético, lo que significa que gran parte de la energía consumida se malgasta. Las pérdidas de energía pueden minimizarse mejorando los edificios ya existentes y apostando por soluciones inteligentes y materiales eficientes desde el punto de vista energético para las nuevas construcciones. 

Además, la mejora de la eficiencia energética de los edificios será determinante para el ambicioso objetivo de conseguir la neutralidad en emisiones de carbono establecido para 2050 en el Pacto Verde Europeo. Por lo tanto, abordar la eficiencia energética en la gestión del IEQ es esencial no solo para el bienestar de los ocupantes, sino también para cumplir con los objetivos climáticos y energéticos de la UE.

\subsection{Desafíos en la resolución del problema} 

La gestión de la calidad ambiental interior (CAI) enfrenta varios retos debido a la complejidad inherente del entorno interior. Estos desafíos surgen tanto de la interacción de múltiples variables como de las limitaciones en las infraestructuras y tecnologías disponibles. A continuación, se detallan los principales obstáculos identificados:

\begin{itemize}
	\item \textbf{Complejidad multidimensional}: las variables que afectan el confort interior están interrelacionadas y pueden presentar comportamientos no lineales, lo que dificulta su modelado y control. Dicha complejidad añadida exige una solución integral que permita gestionar simultáneamente todas las variables interdependientes.
	
	\item \textbf{Subjetividad del confort}: el confort es altamente subjetivo y varía entre individuos, lo que complica la creación de un sistema que satisfaga a todos los usuarios. <<Los conceptos de seguridad, limpieza y aislamiento... abarcan mucho más que la concentración de sustancias respirables y no son universales>> \parencite{vargas2005calidad}.
	
	\item \textbf{Condiciones cambiantes}: factores como el clima exterior, la ocupación del espacio y la actividad de los usuarios pueden cambiar constantemente, requiriendo un sistema flexible y adaptativo.
	
	\item \textbf{Optimización energética}: <<El mantenimiento de las condiciones ambientales interiores óptimas se consigue en gran medida a expensas del aumento en el consumo energético>> \parencite{vargas2005calidad}. Esto viene a indicar que existe una tensión inherente entre mejorar el confort y reducir el consumo de energía, y diseñar un sistema que logre ambos objetivos simultáneamente es un reto significativo.
	
	\item \textbf{Integración tecnológica}: incorporar nuevos sistemas de control en infraestructuras existentes sin interrumpir su funcionamiento o requerir grandes inversiones es otro desafío técnico y económico.
	
\end{itemize}

Estos desafíos subrayan la necesidad de una solución innovadora que pueda abordar la complejidad y dinámica del problema de manera eficiente y efectiva.




